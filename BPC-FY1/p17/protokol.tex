\documentclass{praktikum}


%====== Vyplňte údaje ======
%% dobrá rada na začátek - pokud nevíš, co měníš, zálohuj to, co funguje ;)

\jmeno{Matyáš Peroutík}
\kod{256371}
\rocnik{2023/2024}
\obor{AMT}
\skupina{}
\labskup{B}
\spolupracoval{Štěpán Pavlica}


\ucitel{}
\merenodne{28.\,2.\,2024}
\odevzdanodne{13.\,2.\,2024}

\priprava{}
\opravy{}
\nazev{Vlastnosti ručkových měřících přístrojů}
\cislo{17} %měřené úlohy



\begin{document}
%====== Vygenerování tabulky ======
\maketitle
\vspace{0.5 cm}

%====== Text protokolu zde ======

\section{Úkol měření}
\paragraph{}
Zobrazte na osciloskopu a změřte zadané hodnoty napětí s harmonickým průběhem, a to neusměrněné a jednocestně i dvoucestně usměrněné. Využijte podle možnosti všechny voltmetry u úlohy.
\section{Teoretický rozbor}


\end{document}

