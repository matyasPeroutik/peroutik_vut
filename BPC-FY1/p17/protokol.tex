\documentclass{praktikum}


%====== Vyplňte údaje ======
%% dobrá rada na začátek - pokud nevíš, co měníš, zálohuj to, co funguje ;)

\jmeno{Matyáš Peroutík}
\kod{256371}
\rocnik{2023/2024}
\obor{AMT}
\skupina{}
\labskup{B}
\spolupracoval{Štěpán Pavlica}


\ucitel{}
\merenodne{28.\,2.\,2024}
\odevzdanodne{13.\,2.\,2024}

\priprava{}
\opravy{}
\nazev{Vlastnosti ručkových měřících přístrojů}
\cislo{17} %měřené úlohy



\begin{document}
%====== Vygenerování tabulky ======
\maketitle
\vspace{0.5 cm}

%====== Text protokolu zde ======

\section{Úkol měření}
\paragraph{}
Zobrazte na osciloskopu a změřte zadané hodnoty napětí s harmonickým průběhem, a to neusměrněné a jednocestně i dvoucestně usměrněné. Využijte podle možnosti všechny voltmetry u úlohy.
\section{Teoretický rozbor}

\subsection{Efektivní hodnota elektrických veličin}
Efektivní hodnota elektrické veličiny je hodnota stejnosměrné stálé veličiny, která by za dobu jedné periody signálu na stejném ideálním rezistoru vyzářila stejné teplo. Z této definice vyplývají následující vztahy:

\begin{equation}
\label{eqn:u_efective}
U_{ef}=\sqrt{\frac{1}{T}\int_{0}^{T}u^2(t)dt}\
\end{equation}
\begin{equation}
\label{eqn:i_effective}
I_{ef}=\sqrt{\frac{1}{T}\int_{0}^{T}i^2(t)dt}
\end{equation}

kde u(t) je okamžitá hodnota napětí a i(t) je okamžitá hodnota proudu.

\subsection{Střední hodnota elektrických veličin}
Efektivní hodnota elektrické veličiny je hodnota stejnosměrné stálé veličiny, která by za dobu jedné periody signálu umožnila přenos stejně velkého náboje. Taky se jí občasně říká stejnosměrná složka signálu. Z definice vyplývají následující vztahy:

\begin{equation}
\label{u_meanval}
U_{s}=\frac{1}{T}\int_{0}^{T}u^2(t)dt\\
\end{equation}
\begin{equation}
\label{i_meanval}
I_{s}=\frac{1}{T}\int_{0}^{T}i^2(t)dt\\
\end{equation}

kde u(t) je okamžitá hodnota napětí a i(t) je okamžitá hodnota proudu.

\end{document}

